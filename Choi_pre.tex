\documentclass{beamer}

\mode<presentation>
{
  \usetheme{Madrid}      % or try Darmstadt, Madrid, Warsaw, ...
  \usecolortheme{default} % or try albatross, beaver, crane, ...
  \usefonttheme{serif}  % or try serif, structurebold, ...
  \setbeamertemplate{navigation symbols}{}
  \setbeamertemplate{caption}[numbered]
} 
 
\usepackage[utf8]{inputenc}
\usepackage[]{algorithm2e}

%Information to be included in the title page:
\title{ Learning Low-Dimensional Representations of Medical Concepts}
\author{YD Choi, Yi-I Chiu, David Sontag}
\institute{CILVR Lab, New York University}
 
\begin{document}
 
\frame{\titlepage} 

\begin{frame}
\frametitle{The Article under Investigation}
\begin{table}[h!]
\tbl{Display of a sub-computation for $\text{MCSM}_{\text{ULMS}}(\text{MCECN},\text{Neoplastic Process}, 8)$. The
sub-computation concerns the neighborhood of the medical concept $4003436$(Carcinoma, non-small-cell lung). The
medical concept type annotations are shown in the square brackets. The 
numerical values represent the cosine distance measure of corresponding medical concepts.}
{
\begin{tabular}{|c|} 
\hline
Neighbors of CUI 4003436 (Carcinoma, non-small-cell lung) ['Neoplastic Process'] 
\T \B \\ 
\hline
\textbf{4069419 (small cell carcinoma of lung, C0149925,  ['Neoplastic Process']) : 0.955599426233} \T \B \\
\textbf{4394316 (carcinoma of lung, C0684249,  ['Neoplastic Process']) : 0.933888909808} 
\T \B \\
\textbf{4125384 (malignant neoplasm of lung, C0242379,  ['Neoplastic Process']) : 0.928970432186} \T \B \\
\textbf{4070138 (adenocarcinoma of lung (disorder), C0152013,  ['Neoplastic Process']) : 0.924754262378} \T \B \\
4555365 (tarceva, C1135136,  ['Organic Chemical', 'Pharmacologic Substance']) : 0.917757636841 \T \B \\
4069342 (lung mass, C0149726,  ['Finding']) : 0.914073299934 \T \B \\
4542086 (alimta, C1101816,  ['Organic Chemical', 'Pharmacologic Substance']) : 0.903354063605 \T \B \\
\textbf{4148168 (non-small cell lung cancer metastatic, C0278987,  ['Neoplastic Process']) : 0.899565393859} \T \B \\
\hline
\end{tabular}
}
\end{frame}

\begin{frame}
\frametitle{Genetics meets Surgical Technologies: \\
CRISPR and Xenotransplantation}
\begin{itemize}
\item CRISPR : A recently developed method for 
``editing genes."
\item Xenotransplantation :
The transplantation of living cells, tissues or organs
from one species to another. 
\item It has been recently shown that a particular
complication that arises in
xenotransplantation, using pig organs,
can be solved through gene-editing via CRISPR. 
\end{itemize}
\end{frame}

\begin{frame}
\frametitle{Development from Genetics: CRISPR}
\begin{itemize}
\item In October of 2015, scientists gathered at the National Academy of Sciences
in Washington to talk about CRISPR, a new method for editing genes.

\item Carl Zimmer claims that ``In the past couple of years, 
the technique has become so powerful and accessible that many experts are 
calling for limits on its potential uses — especially altering human 
embryos with changes that could be inherited by future generations."
\end{itemize}
\end{frame}



\end{document}
